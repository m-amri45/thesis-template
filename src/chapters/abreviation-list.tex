\clearpage
\chapter*{DAFTAR SINGKATAN DAN LAMBANG}
\addcontentsline{toc}{chapter}{DAFTAR SINGKATAN DAN LAMBANG}


\begin{table}[ht]
	\centering
	\begin{tabularx}{\textwidth}{>{\raggedright\arraybackslash}X >{\raggedright\arraybackslash}p{8cm} >{\centering\arraybackslash}X}
		SINGKATAN & \multicolumn{1}{c}{Nama}                  & \multicolumn{1}{>{\raggedright\arraybackslash}X}{Pemakaian pertama kali pada halaman} \\
		AC1       & \textit{Acronym1}                         & 1                                                                                     \\
		AC2       & \textit{Acronym2}                         & 1                                                                                     \\                                                                                    \\
	\end{tabularx}
\end{table}

\begin{table}[ht]
	\centering
	\begin{tabularx}{\textwidth}{>{\raggedright\arraybackslash}X >{\raggedright\arraybackslash}p{8cm} >{\centering\arraybackslash}X}
		LAMBANG  & \multicolumn{1}{c}{Arti}                                      & \multicolumn{1}{>{\raggedright\arraybackslash}X}{Pemakaian pertama kali pada halaman} \\
		$A$      & Set aksi \textit{reinforcement learning}                      & 6                                                                                     \\
		$S$      & Set \textit{state} \textit{reinforcement learning}            & 6                                                                                     \\
		$\pi*$   & Strategi optimal \textit{reinforcement learning}              & 6                                                                                     \\
		$\gamma$ & Konstanta diskon pembelajaran \textit{reinforcement learning} & 6                                                                                     \\
		$\alpha$ & Konstanta pembelajaran \textit{reinforcement learning}        & 6                                                                                     \\
		$V$      & Fungsi nilai \textit{reinforcement learning}                  & 6                                                                                     \\
		$Q$      & Fungsi \textit{Q-Table}                                       & 7                                                                                     \\
		$\delta$ & Konstanta pembanding pada algoritma memoisasi                 & 23                                                                                    \\
	\end{tabularx}
\end{table}


\begin{acronym}
	\acro{AC1}{\textit{Acronym1}}
	\acro{AC2}{\textit{Acronym2}}
\end{acronym}

\clearpage
